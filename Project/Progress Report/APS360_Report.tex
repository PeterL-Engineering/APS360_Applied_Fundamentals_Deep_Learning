\documentclass{article} % For LaTeX2e
\usepackage{iclr2022_conference,times}
% Optional math commands from https://github.com/goodfeli/dlbook_notation.
\input{math_commands.tex}

%######## APS360: Uncomment your submission name
%\newcommand{\apsname}{Project Proposal}
\newcommand{\apsname}{Progress Report}
%\newcommand{\apsname}{Final Report}

%######## APS360: Put your Group Number here
\newcommand{\gpnumber}{40}

\usepackage{hyperref}
\usepackage{xcolor}
\usepackage[normalem]{ulem}
\usepackage{url}
\usepackage{graphicx}

%######## APS360: Put your project Title here
\title{Image Colourization via Convolutional \\
Neural Networks and Deep Learning}


%######## APS360: Put your names, student IDs and Emails here
\author{Youssef Fikry  \\
Student\# 1005678901\\
\texttt{youssef.fikry@mail.utoronto.ca} \\
\And Harkirpa Kaur  \\
Student\# 1011242479 \\
\texttt{harkirpa.kaur@mail.utoronto.ca} \\
\AND
Peter Leong \\
Student\# 1005678901 \\
\texttt{peter.leong@mail.utoronto.ca} \\
\And
Thulasi Thavarajah \\
Student\# 10115358424 \\
\texttt{t.thavarajah@mail.utoronto.ca} \\
\AND
}

% The \author macro works with any number of authors. There are two commands
% used to separate the names and addresses of multiple authors: \And and \AND.
%
% Using \And between authors leaves it to \LaTeX{} to determine where to break
% the lines. Using \AND forces a linebreak at that point. So, if \LaTeX{}
% puts 3 of 4 authors names on the first line, and the last on the second
% line, try using \AND instead of \And before the third author name.

\newcommand{\fix}{\marginpar{FIX}}
\newcommand{\new}{\marginpar{NEW}}

\iclrfinalcopy 
%######## APS360: Document starts here
\begin{document}


\maketitle

\begin{abstract}
This project addresses the challenge of automated colourization for 256$\times$256 grayscale images using a dataset of 12,600 image pairs, balanced across human subjects, 
animals, and natural scenery. We frame colourization as a supervised learning problem in the CIELAB colour space, where a model predicts chrominance channels ($a^*$, $b^*$) 
from the luminance channel ($L^*$). A shallow convolutional neural network (CNN) provides the baseline performance, while our primary solution employs a deeper convolutional encoder-decoder architecture. This 
design captures high-level semantic features and spatial context, addressing limitations of shallow networks in perceptual realism. All source code, datasets, and results 
are publicly available \href{https://drive.google.com/drive/folders/1cV1NhlQ8UTk_CgJdwhqeRu0z5xE85ZsI?usp=sharing}{here}. 
%######## APS360: Do not change the next line. This shows your Main body page count.
----Total Pages: \pageref{last_page}
\end{abstract}


\section{Introduction}

While colour photography processes first emerged in the 1890s, widespread accessibility was not achieved until the 1970s \citep{scienceandmediamuseum2020}. Consequently, most historic
photographs exist only in black and white, lacking the visual richness expected by modern audiences. For individuals with vision impairments---such as those who have undergone cataract 
surgery---interpreting grayscale images poses significant challenges \citep{vogelsang2024impact}, rendering much of photographic history inaccessible to them. 

This project leverages deep learning to automate the colourization of black-and-white images, with the dual goals of restoring lost visual information and enhancing accessibility. 
Traditional colourization methods often yield desaturated results and require laborious manual intervention \citep{cheng2016deepcolorization}, whereas convolutional neural networks (CNNs) 
learn spatial and semantic features to generate realistic colours autonomously \citep{zhang2016colorful}. By automating this process, we aim to make historical imagery more engaging and inclusive.

\section{Inidividual Contributions & Responisbilites}

\section{Notable Contributions}

\subsection{Data Processing}

\subsection{Baseline Model}

\subsection{Primary Model}
\label{last_page}

\newpage
\bibliographystyle{iclr2022_conference}
\bibliography{APS360_Report_ref}

\end{document}
