\documentclass{article} % For LaTex2e
\usepackage{iclr2022_conference,times}
% Optional math commands from https://github.com/goodfeli/dlbook_notation.
\input{math_commands.tex}

%######## APS360: Uncomment your submission name
%\newcommand{\apsname}{Project Proposal}
%\newcommand{\apsname}{Progress Report}
\newcommand{\apsname}{Final Report}

%######## APS360: Put your Group Number here
\newcommand{\gpnumber}{40}

\usepackage{hyperref}
\usepackage{xcolor}
\usepackage[normalem]{ulem}
\usepackage{url}
\usepackage{graphicx}
\usepackage{placeins}
\usepackage{float}

%######## APS360: Put your project Title here
\title{Image Colourization via Convolutional \\
Neural Networks and Deep Learning}

%######## APS360: Put your names, student IDs and Emails here
\author{Youssef Fikry  \\
Student\# 1006682626\\
\texttt{youssef.fikry@mail.utoronto.ca} \\
\And Harkirpa Kaur  \\
Student\# 1011242479 \\
\texttt{harkirpa.kaur@mail.utoronto.ca} \\
\AND
Peter Leong \\
Student\# 1010892955 \\
\texttt{peter.leong@mail.utoronto.ca} \\
\And
Thulasi Thavarajah \\
Student\# 10115358424 \\
\texttt{t.thavarajah@mail.utoronto.ca} \\
\AND
}

% The \author macro works with any number of authors. There are two commands
% used to separate the names and addresses of multiple authors: \And and \AND.
%
% Using \And between authors leaves it to \LaTex{} to determine where to break
% the lines. Using \AND forces a linebreak at that point. So, if \LaTex{}
% puts 3 of 4 authors names on the first line, and the last on the second
% line, try using \AND instead of \And before the third author name.

\newcommand{\fix}{\marginpar{FIx}}
\newcommand{\new}{\marginpar{NEW}}

\iclrfinalcopy 
%######## APS360: Document starts here
\begin{document}


\maketitle

\begin{abstract}
This project addresses the challenge of automated colourization for 256$\times$256 grayscale images using a dataset of 12,600 image pairs, balanced across human subjects, 
animals, and natural scenery. We frame colourization as a supervised learning problem in the CIELAB colour space, where a model predicts chrominance channels ($a^*$, $b^*$) 
from the luminance channel ($L^*$). A shallow convolutional neural network (CNN) provides the baseline performance, while our primary solution employs a deeper convolutional 
encoder-decoder architecture. This design captures high-level semantic features and spatial context, addressing limitations of shallow networks in perceptual realism. All 
source code, datasets, and results are publicly available \href{https://drive.google.com/drive/folders/1cV1NhlQ8UTk_CgJdwhqeRu0z5xE85ZsI?usp=sharing}{here}. 
%######## APS360: Do not change the next line. This shows your Main body page count.
----Total Pages: \pageref{last_page}
\end{abstract}


\section{Introduction}

While colour photography processes first emerged in the 1890s, colour photography did not become widely accessible until the 1970s \citep{scienceandmediamuseum2020}. 
Consequently, most historical photographs remain in black and white, lacking the visual richness that modern viewers are accustomed to. Moreover, individuals who undergo cataract 
removal as part of vision restoration procedures often struggle to interpret grayscale images, rendering many historical photographs inaccessible to them \citet{vogelsang2024impact}. 
This project aims to leverage deep learning to automatically colourize black and white images, with the goal of restoring visual information and improving accessibility for all audiences. 
Traditional, non-deep learning colourization methods tend to produce desaturated results and require extensive human input, limiting their scalability \citep{cheng2016deepcolorization}. 
In contrast, deep neural networks such as convolutional neural networks (CNNs) can effectively learn spatial and semantic features, enabling realistic colourization without user 
intervention \citep{zhang2016colorful}. This makes deep learning a promising and scalable solution for image colourization.

\section{Contributions}

\begin{table}[h]
\centering
\begin{tabular}{cccc}
\hline
Harkirpa & Peter & Thulasi & Youssef \\
24\% & 23\% & 26\% & 27\% \\
\hline
\end{tabular}
\end{table}

Throughout the project, I was responsible for several important tasks that helped move the work forward. I conducted all the preliminary background research needed for the proposal, 
which helped define the project's scope and goals. I also helped create the project's Gantt chart to manage the timeline and track key milestones. One of my main contributions was designing 
and implementing the baseline colourization model, including coding and debugging. I wrote a large part of the progress report to clearly document our work and challenges.

I researched various colourization methods to explore ways to improve the model and implemented many of these techniques to add new features. I completed most of the final report, explaining 
our results and progress. I also prepared most of the content for the final presentation, focusing on communicating our work clearly. Additionally, I regularly helped organize and document the 
minutes from weekly team meetings to keep track of decisions and discussions. Throughout the project, I worked closely with my teammates to support our development process.

Some tasks were not completed as planned. Although I researched many diverse colourization methods, the full implementation of all these techniques was not finished on time. Some debugging 
and polishing of advanced methods remain unfinished.

\label{last_page}

\newpage
\bibliographystyle{iclr2022_conference}
\bibliography{APS360_Final_Report_Solo_ref}


\end{document}
